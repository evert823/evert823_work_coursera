\documentclass[12pt, letterpaper]{article}
\usepackage{graphicx}
\usepackage{amsmath}
\usepackage{amssymb}
\usepackage{float}
\graphicspath{{images/}}
\title{RSS and the Gradient in Matrix form - with Ridge}
\author{Evert Jan Karman}
\date{January 1900}
\begin{document}
\maketitle

Recap: The RSS and the Gradient in matrix form - with Ridge

\section{The formula for RSS}
\[
\mathrm{RSS(w)} = \left(y - H w\right)^{\top} \left(y - H w\right) + \lambda w^{\top} w
\]

\section{The formula for the Gradient}
\[
\nabla\, \mathrm{RSS(w)} = -2 H^{\top} \left(y - H w\right) + 2 \lambda w
\]

\section{The formula for RSS with NO penalty for intercept (w0)}
\[
\mathrm{RSS(w)} = \left(y - H w\right)^{\top} \left(y - H w\right) + \lambda \sum_{j=1}^{D-1} w_{j}^{2}
\]
(that is assuming that we have D features and w0 is our intercept)

If we wanted to write this in matrix form, let
\[
P=
\begin{pmatrix}
0 & 0 & 0 & \cdots & 0 \\
0 & 1 & 0 & \cdots & 0 \\
0 & 0 & 1 & \cdots & 0 \\
\vdots & \vdots & \vdots & \ddots & \vdots \\
0 & 0 & 0 & \cdots & 1
\end{pmatrix}
\]
and
\[
\mathrm{RSS(w)} = \left(y - H w\right)^{\top} \left(y - H w\right) + \lambda w^{\top} P w
\]


\section{The formula for the Gradient with NO penalty for intercept (w0)}
\[
\nabla\, \mathrm{RSS(w)} = -2 H^{\top} \left(y - H w\right) + 2 \lambda
\begin{pmatrix}
0\\[4pt]
w_1\\[4pt]
\vdots\\[4pt]
w_{D-1}
\end{pmatrix}
\]

or in matrix form
\[
\nabla\, \mathrm{RSS(w)} = -2 H^{\top} \left(y - H w\right) + 2 \lambda P w
\]


\end{document}
